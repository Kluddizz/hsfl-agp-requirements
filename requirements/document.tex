\documentclass{article}
\usepackage[ngerman]{babel}
\usepackage{enumitem}

\title{
  \Huge Advanced Game Programming \\
  \vspace{0.4cm}
  \large Prozedurale Sterne: Anforderungen
}

\author{
  Florian Hansen \and
  Markus Behnisch \and
  Julian Kasiske
}

\begin{document}
  \maketitle

  \section{Funktionale Anforderungen}
  \setlist[enumerate]{label*={\arabic*.}}
  \begin{enumerate}

    \vspace{0.5cm}

    {\bfseries\large\item Eckpunkte einer Kugel}
    \begin{enumerate}
      \item Die einfache Implementierung von Unity (und anderen Tools) einer
        Kugel besitzt eine Überabtastung in den Polen, sodass ein anderer
        Algorithmus zum Erzeugen verwendet werden soll.
      \item Es soll ein Algorithmus in Unity implementiert werden, welcher einen
        Icosahedron mit einem individuellen Level-Of-Detail erzeugt.
      \item Es sollen \texttt{vertex}-, \texttt{index}- und
        \texttt{uv}-Koordinaten berechnet werden.
      \item Es soll ein Benutzerelement erstellt werden, welches den Algorithmus
        ausführt und das entsprechende Mesh erzeugt.
    \end{enumerate}

    \vspace{0.5cm}

    {\bfseries\large\item Textur der Sonnenoberfläche}
    \begin{enumerate}
      \item Es soll ein Algorithmus implementiert werden, welcher die
        Sonnenoberfläche prozedural darstellen soll.
      \item Die grundlegende Textur soll mithilfe eines
        Cellular-Noise-Al\-go\-rith\-mus' generiert werden.
      \item Es soll der Combustible-Voronoi-Algorithmus' verwendet
        werden, um das "Plasma" der Sonne darzustellen.
    \end{enumerate}

    \vspace{0.5cm}

    {\bfseries\large\item Deformation der Sonnenoberfläche}
    \begin{enumerate}
      \item Die Oberfläche der Sonne besteht aus Plasma, welches sich
        wellenartig bewegt.
      \item Der zu entwickelnde Shader soll die Oberfläche der Kugel
        manpulieren, sodass sich diese ähnlich wie Wasser verhält.
    \end{enumerate}

    \vspace{0.5cm}

    {\bfseries\large\item Sonneneruptionen}
    \begin{enumerate}
      \item Es sollen
        Sonnenstürme\footnote{https://solarscience.msfc.nasa.gov/flares.shtml}
        (Eruptionen) implementiert werden, die aus der Sonnenoberfläche
        schießen.
      \item Eruptionen sind Explosionen auf der Sonnenoberfläche und werden
        durch magnetische Felder erzeugt. Es sind also einfache Explosionen, die
        durch Partikelsysteme dargestellt werden sollen.
      \item Diese ausgesendeten Partikel einer Explosion sollen dann entlang des
        magnetischen Feldes (vom Pluspol zum Minuspol) wandern.
      \item Die magnetische Laufbahn der Partikel soll zur Vereinfachung als
        Parabel angenommen werden.
    \end{enumerate}

  \end{enumerate}
\end{document}
