\documentclass{article}
\usepackage[ngerman]{babel}
\usepackage{enumitem}
\usepackage{graphicx}
\usepackage{placeins}

\title{
  \Huge Advanced Game Programming \\
  \vspace{0.4cm}
  \large Prozedurale Sterne: Anforderungen
}

\author{
  Florian Hansen \and
  Markus Behnisch \and
  Julian Kasiske
}

\begin{document}
  \maketitle

  \section{Funktionale Anforderungen}
  \setlist[enumerate]{label*={\arabic*.}}
  \begin{enumerate}

    \vspace{0.5cm}

    {\bfseries\large\item Generierung einer homogenen Kugel}
    \begin{enumerate}
      \item Die einfache Implementierung von Unity (und anderen Tools) einer
        Kugel besitzt eine Überabtastung an den Polen, sodass ein anderer
        Algorithmus zum Erzeugen verwendet werden soll.
      \item Es soll ein Algorithmus in Unity implementiert werden, welcher eine
        Icosphere mit einem beliebigen Level-Of-Detail erzeugt.
      \item Es sollen \textit{Vertices}, \textit{Indices} und
        \textit{UV}-Koordinaten berechnet werden.
      \item Es soll ein Skript implementiert werden, welches den Algorithmus
        ausführt und das entsprechende Mesh erzeugt.
      \item Es sollen entsprechende Steuerelemente (Attribute) implementiert
        werden, um die Kugel zu verändern. Z.B. soll der Level-Of-Detail
        mithilfe eines Steuerelementes interaktiv verändert werden können.
      \item Es sollen entsprechende \textit{Gizmos} implementiert werden, um die
        Kugel auch im Editor von Unity sehen zu können.
      \item Transformationen des Kugel-Objektes sollen sich ebenfalls auf die
        Position der Gizmos auswirken.
    \end{enumerate}

    \vspace{0.5cm}

    {\bfseries\large\item Textur der Sonnenoberfläche}
    \begin{enumerate}
      \item Für die Textur der Sonnenoberfläche soll ein Algorithmus
        implementiert werden.
      \item Der Algorithmus soll innerhalb des \textit{Fragment-Shaders}
        ausgeführt werden.
      \item Die Textur soll prozedural erzeugt werden. Das bedeutet, dass diese sich
        abhängig von der Zeit verändert und nicht als statische Textur geladen
        werden soll.
      \item Die Parameter der Textur sollen durch interaktive Elemente
        verändert werden können.
      \item Das Erscheinungsbild der Textur soll der tatsächlichen Oberfläche
        der Sonne ähneln. Dabei ist es wichtig zu beachten, dass sich die
        relative Entfernung zur Sonne auf das Erscheinungsbild auswirkt.
      \item Ist die Kamera der Szene weit von der Sonne entfernt, soll eine
        sturmartige Textur angezeigt werden. Bei naher Betrachtung sollen
        zellenartige Strukturen sichtbar werden.
    \end{enumerate}

    \vspace{0.5cm}

    {\bfseries\large\item Deformation der Sonnenoberfläche}
    \begin{enumerate}
      \item Die Oberfläche der Sonne besteht aus Plasma, welches sich
        wellenartig bewegt.
      \item Der zu entwickelnde \textit{Vertex-Shader} soll die Oberfläche der
        Kugel manipulieren, sodass diese nicht starr ist.
      \item Die Manipulation soll sich nach der Textur der Sonnenoberfläche
        richten. Dunkle Stellen der Oberfläche sollen näher an dem Zentrum der
        Kugel sein, wobei helle (heiße) Stellen aufgrund ihrer Dynamik weiter
        entfernt sein sollen.
      \item Jeder Punkt der Kugeloberfläche soll eine Temperatur bekommen,
        sodass sich heiße Punkte schneller bewegen als kalte.
      \item Die Temperaturinformation soll sich ebenfalls aus der Textur
        ergeben.
    \end{enumerate}

    \vspace{0.5cm}

    {\bfseries\large\item Protuberanzen}
    \begin{enumerate}
      \item Es sollen
        Protuberanzen\footnote{https://de.wikipedia.org/wiki/Protuberanz}
        implementiert werden, die aus der Sonnenoberfläche schießen.
      \item Es sollen ruhende Protuberanzen implementiert werden.
      \begin{enumerate}
        \item Diese sollen sich über die Zeit kaum verändern, das heißt, die
          Magnetfeldlinien, die sie erzeugen verändern sich nicht.
        \item Sie sollen dunkler dargestellt werden, als die Umgebung, da sie
          durch ihre Starrheit abkühlen.
      \end{enumerate}
      \item Es sollen eruptive Protuberanzen (Sonneneruptionen) implementiert werden.
      \begin{enumerate}
        \item Eruptionen sind Explosionen auf der Sonnenoberfläche und werden
          durch magnetische Felder erzeugt. Es sind also einfache Explosionen, die
          durch Partikelsysteme dargestellt werden sollen.
      \end{enumerate}
    \end{enumerate}
  \end{enumerate}

\end{document}
